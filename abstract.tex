%\renewcommand\abstractname{摘~要}
\begin{abstract}

熔盐堆实验(MSRE)的运行证实了在反应堆候选者中氟盐混合物作为实际的流体核燃料是相当稳定的。%
然而,有些裂变产物化学物质离开了核燃料\index{核燃料}的循环系统,出现在堆芯石墨慢化剂、和熔盐接触的金属组件表面、排气系统的金属组件表面。%
Xe和Kr裂变气体在排气系统里衰变成其他子体,并被分离出来。%
有些元素(Mo,Nb,Ru,Te,Sb)以金属形式存在,并附着在金属组件表面、石墨表面、或者以粒子的形式被带到排气系统。%
在设计更大的熔盐堆系统中,已知裂变产物以多少比例分布在哪些部位是相当重要的,从MSRE实验中获得信息,并制定出值得考虑的影响因素。

在MSRE实验中,通过测量发射出的 \(\gamma\) 射线的强度和能谱,ORNL发展了一种技术来定位、测量与熔盐接触的表面或排气系统中的裂变产物的沉积情况。%
发展出来的\(\gamma\)谱测量装置包括一个Ge(Li)探头,4096道的多道分析器,和一个用来测量小区域的铅准直器。%
这套装置经常安装在MSRE可移动的维护屏上面,这些维护屏覆盖在堆系统组件的各个部位,通过激光束配合观察员的搬运,可以实现装置的精确准直和固定。

本实验不仅在停堆和排出熔盐时,也在核燃料循环和堆运行在不同功率等级下做测量。总共记录1000多个谱,其中,堆运行在不同功率等级下的记录的谱占了25\%。%
另外,有400个谱用于刻度仪器。计算机化数据处理实现了高质量、大批量的数据的分析能力。

本实验主要的精力集中在排气系统和主热交换器,其中,后者有40\%的金属表面与熔盐接触。%
排气系统不仅包含了气态裂变产物及其母体,还包括金属元素及其衰变产物(例如,Nb,Mo,Ru,Sb,Te(I))。%
MSRE的热交换器主要沉积类似的金属元素。%
当停堆且立即排出燃料(紧急排放)时,实验观察到热交换器的放射性主要来源于裂变气体。

在高放射性堆系统中定位和评估裂变产物的沉积情况,使用了远程维护设备和定位工具的高分辨率$\gamma$射线谱测量法被证实是非常有用的。

\end{abstract}
