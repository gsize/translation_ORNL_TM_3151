\chapter{研究γ谱测量法的目的}
MSRE的运行证明了氟盐混合物在堆中是稳定的,且大部分裂变产物都保持在循环的核燃料盐中;然而,在堆芯石墨慢化剂、与熔盐接触的金属组件表面和堆排气系统中发现了一些裂变产物。例如,Mo、Nb、Ru、Te、Sb等元素以金属单质的形式存在,并逐渐附着在与熔盐接触的组件上面,或以粒子形式被携带到排气系统中。

在熔盐系统中,某些裂变产物(特别是易挥发或易沉积的)的行为由于以下几个原因而被受关注:
\begin{enumerate}
\item 为了了解熔盐中裂变产物的化学性质。
\item 堆组件的远程维护所需的屏蔽由大量沉积在这些组件上的高放射性裂变产物决定。
\item 对于一个高功率的熔盐堆而言,在停堆和排出燃料之后,冷却这些沉积的裂变产物所释放的几兆瓦特的衰变热将是一个麻烦。
\item 沉积堆积在堆芯石墨里的裂变产物将吸收更多的中子,因此降低熔盐堆的增值性能。
\end{enumerate}
由以上原因,MSRE承担了裂变产物行为研究的整个计划。这篇报告要描述的研究目的是通过远程 $\gamma$ 能谱测量技术对沉积在某些MSRE组件上的放射性裂变产物进行定性、定量测量。尤其关注反应堆的排气系统和热交换器,其中热交换器大约有40\%的金属表面和循环的熔盐核燃料接触。堆中石墨、金属组件和泵槽的沉积物正由其他团队研究并有另外报告加以讨论\footnote{F. F. Blankendhip et al.,Msr Program Semiannu. Progr. Rep. Aug. 31, 1969, ORNL-4449, pp.104-9} \footnote{C. H. Gabbard, MSR Program Semiannu. Progr. Rep. Feb. 28. 1970, ORNL-4548, pp.13.} \footnote{F. F. Blankendhip et al., ibid., pp.104-8.} \footnote{F. F. Blankendhip et al.,Msr Program Semiannu. Progr. Rep. Aug. 31, 1970, ORNL-4622, pp.60-70.}。本报告通过一些说明以一种现实有用的形式介绍了远程 $\gamma$ 能谱测量的结果,这些说明对了解MSRE实验中的裂变产物行为的整体影响可能有些帮助。

