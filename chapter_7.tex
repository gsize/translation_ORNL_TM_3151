\chapter{结果}

本章根据上一章列出的测量谱排序,详细描述MSRE中裂变产物的分布情况。由于谱图数量多,其结果主要以图表的形式呈现,对于一些明显错误的数据,将直接舍弃。

\section{A组谱}

A组谱是在停堆期间时记录的,采谱点为换热器,主排气线,熔盐回路。大多数谱是在停堆6周后获得(1969.6.1),意味着只有一些长寿命的核素存在,谱图也相对简洁,无重峰现象。谱的采集时间约为200秒。有多条$\gamma$\ 射线的核素,只用比较显著的光电峰得出的活度的均值作为结果。

虽然$^{137}$Cs和$^{140}$Ba—La多次在换热器中被检测到,但它们的活度太低,推测是在停堆排空后,系统中的Xe衰变后沉积的结果。$^{95}$Zr未被检测到。本段列出的结果由指数倒推到停堆时刻计算得出的,倒推的结论可以与不同时刻采谱得出的结果相比较,然而倒推值由于母体衰变的影响而偏高。如$^{131}$I倒推值由于$^{131m}$Te的衰变而偏大。

\subsection{换热器}

沿换热器的纵向采得65个谱,水平方向四个点上采集49个谱。由于换热器的屏蔽状态的变化,如换热管数目的变化,光子通过换热器时的强度衰减,无法对水平方向进行定量分析。而裂变产物在纵向上是对称分布的。在换热器连接盒(HTR plug)处采集的谱图由于涉及未知的屏蔽参数,而无法做定量分析。此处的结果远低于其他谱,这不是由于裂变产物的不同沉积引起的。

\underline{Ni-95 \ref{fig_7_1}} 沿换热管纵向的衰变率在$0.1\sim 0.3\times 10^{12}dis/min/cm^2$,在阻挡板附近,值为$0.25\sim 0.3\times 10^{12}dis/min/cm^2$,在阻挡板之间的值$0.1\sim 0.15\times 10^{12}dis/min/cm^2$位于壳上侧的阻挡板的活度明显增大,推测是由于靠近挡板和壳位置的流速较低引起的。由图似乎观察不出活度沿着换热器纵向存在规律性变化。

\underline{Ru-103} 沿着纵向的衰变率几乎与$^{95}$Nb一样,值在$0.17\sim 0.65\times 10^{11}dis/min/cm^2$左右,在挡板附近为$0.5\sim 0.65\times 10^{11}dis/min/cm^2$之间,而在挡板之间的为$0.17\sim 0.24\times 10^{11}dis/min/cm^2$。

\underline{Ru-Rh-106} $^{106}$Ru是通过其短寿命子体$^{106}$Rh辩别出来,其衰变率和$^{103}$Ru相当。

\underline{Sb-125} 在换热器中观测到少量的存在,由于其偏低的裂变产額,其活度较低($0.053\sim 0.14\times 10^{10}dis/min/cm^2$)。

\underline{Sb-126 Sb-127}

\subsection{热交换器}

\subsection{堆芯废气线}

\subsection{热交换器}
